\input{preambulo.tex}
\hyphenpenalty10000

\begin{document}

\section{Formato de los mensajes DHCP} % (fold)
\label{sec:Formato de los mensajes DHCP}

% section Formato de los mensajes DHCP (end)

\subsection{Vistazo general al formato de un mensaje} % (fold)

% subsection Vistazo general al formato de un mensaje (end)

\begin{frame}[fragile]{Origen del formato de mensaje de DHCP}

    Los clientes y servidores se comunican intercambiando mensajes como se
    describe en la especificación del protocolo. Todos los mensajes DHCP
    comparten un formato común que describiremos ahora.\\[0.2cm]

    DHCP fue desarrollado a partir de BOOTP (ver RFC 951) y usa un formato de
    mensaje que se basa en su especificación. Como DHCP comparte los puertos
    UDP 67 y 68 con BOOTP, los mensajes DHCP incluyen una opción especial en el
    campo 'option' que los diferencia de los mensajes BOOTP.\\[0.2cm]

\end{frame} 

\begin{frame}[fragile]{Descripción del formato de mensaje}

    Todos los mensajes DHCP incluyen una sección de formato fijo y una sección
    de formato variable. La sección de formato fijo contiene varios campos que
    siempre están presentes y la parte de formato variable es la que lleva las
    opciones adicionales que no siempre se incluyen (servicios por
    ejemplo).\\[0.2cm]

    La sección de formato fijo y formato de la parte variable cambian
    dependiendo del tipo de mensaje que esté siendo enviado. La sección de
    formato fijo se divide en varios campos que llevan información como:
    \begin{itemize}
        \item La dirección de red y dirección física del cliente
        \item La dirección IP del servidor
        \item Información de control sobre el mensaje
    \end{itemize}

    Estos campos aparecen en todos los mensajes DHCP, a pesar de que no todos
    los campos se usan en todos los tipos de mensaje.

\end{frame}

\begin{frame}[fragile,allowframebreaks]{La sección de formato fijo}

    Esta sección tal y como la veremos en la próxima página, aparece en todos
    los mensajes DHCP. Cada fila representa 32 bits de la sección de formato
    fijo. Los campos individuales están etiquetados con el nombre del campo.
    Algunos de los campos más largos están resumidos en la figura y su longitud
    se describe explicitamente.\\[0.2cm]

    \framebreak
    
    \begin{figure}
    \begin{center}

    \includegraphics[scale=0.50]{images/6-1.png}
    \caption{Diagrama del formato fijo de un mensaje DHCP}
    \label{FIXED-1}

    \end{center}
    \end{figure}


\framebreak

\begin{tabular}{ l p{0.8\textwidth} }
\hline
\textbf{Campo} & \textbf{Descripción} \\ 
\hline
op & Código de operación de mensaje; valor de 1 en los mensajes enviados por un
cliente y 2 en los mensajes enviados por el servidor. Los dos posibles valores
para 'op' son heredados de BOOTP para retro-compatibilidad. \\
\hline
htype & Tipo de dirección física, definiciones tomadas de la lista de la IANA
para tipos de hardware (ver RFC 1700). Por ejemplo, el tipo de 'Ethernet' esta
especificado cuando htype tiene valor 1. \\
\hline
hlen & Longitud de la dirección física en bytes, define la longitud de la
dirección física en el campo 'chaddr' \\
\hline
hops & Cantidad de 'relay agents' que han re-enviado el mensaje \\
\hline
xid & Identificador de transacción; usado por los clientes para relacionar
respuestas de un servidor con las peticiones transmitidas previamente \\
\hline
secs & Tiempo pasado en segundos, desde que el cliente ha comenzado el proceso
DHCP \\ 
\hline
flags & el bit menos significativo, llamado el 'bit de broadcast' puede tener
el valor de '1' para indicar que el mensaje al cliente debe ser enviado en modo
broadcast \\
\end{tabular}

\framebreak

\begin{tabular}{ l p{0.8\textwidth} }
\hline
\textbf{Campo} & \textbf{Descripción} \\ 
\hline
ciaddr & Dirección IP del cliente; establecida por el cliente solo cuando el
cliente ha confirmado que la dirección IP es válida \\
\hline
yiaddr & Dirección IP del cliente; establecida por el servidor para informar al
cliente de su dirección IP (your IP address) \\
\hline
siaddr & La dirección IP del siguiente 'servidor' que el cliente debe consultar
en su proceso de configuración (por ejemplo el servidor a contactar para
descargar un kernel por TFTP) \\
\hline
giaddr & Dirección del 'relay agent' o 'gateway IP address' establecida por el
'relay agent' con la dirección de la interfaz a través de la que el mensaje fue
recibido \\
\hline
chaddr & Dirección física del cliente \\
\hline
sname & Nombre del siguiente servidor que el cliente debe contactar en el
proceso de configuración \\
\hline
file & Nombre del archivo que el cliente debe solicitar del siguiente servidor,
por ejemplo el nombre del archivo que contiene el sistema operativo para este
cliente \\
\hline
\end{tabular}

\end{frame}

\begin{frame}{Opción: tipo de mensaje DHCP}
    
    La opción 'tipo de mensaje' incluída en cada mensaje contiene la opción de código
    53, la longitud del campo (1) y el tipo de mensaje, en un solo byte con uno
    de los siguientes valores:\\[0.2cm]

\begin{tabular}{ l c }
\hline
\textbf{Tipo de mensaje} & \textbf{Valor de opción} \\ 
\hline
DHCPDISCOVER & 1 \\
\hline
DHCPOFFER & 2 \\
\hline
DHCPREQUEST & 3 \\
\hline
DHCPDECLINE & 4 \\
\hline
DHCPACK & 5 \\
\hline
DHCPNAK & 6 \\
\hline
DHCPRELEASE & 7 \\
\hline
DHCPINFORM & 8 \\
\hline
DHCPFORCERENEW & 9 \\
\hline
\end{tabular}

\end{frame}

\subsection{Transmisión de mensajes DHCP} % (fold)

% subsection Transmisión de mensajes DHCP (end)

\begin{frame}[fragile]{Uso de UDP para DHCP}

    El cliente envía todos sus mensajes DHCP al el puerto 67 (puerto por donde
    espera peticiones el servidor) y usa el puerto 68 como puerto de origen y
    usa el puerto 68 como puerto de origen. El cliente usa \textit{broadcast} en caso
    de que aún no tenga una dirección. Si el cliente sabe la dirección de un
    servidor DHCP y tiene asignado un IP, envía el mensaje directamente al
    servidor.\\[0.2cm]

\end{frame}

\begin{frame}[fragile]{Múltiples agentes de relay}

    Un mensaje DHCP puede ser re-enviado por más de un 'relay agent'. En este
    caso el segundo 'relay agent' detecta que el campo 'giaddr' ya tiene un
    valor distinto a 0.0.0.0. Dicho mensaje es reenviado, aumentando el valor
    de encabezado 'hops' sin cambiar 'giaddr'.\\[0.2cm]

    Desde el punto de vista del servidor, al recibir un mensaje que ha sido
    re-enviado por varios 'relay agent', 'giaddr' contiene el IP del primer
    'relay agent', así el servidor responde de manera directamente a la
    dirección especificada allí, saltándose cualquier 'relay agent'
    intermedio.\\[0.2cm]

\end{frame}

\begin{frame}[fragile]{Entrega confiable de mensajes}

   Como UDP no garantiza la entrega de mensajes, DHCP se encarga de garantizar
   la entrega confiable. Un cliente usa la respuesta de un servidor como una
   notificación implícita de que se recibió el mensaje. Si el cliente
   no recibe una respuesta, retransmite el mensaje hasta que
   reciba una respuesta, o hasta que decida que el servidor no va a
   responder.\\[0.2cm]

   La especificación de DHCP establece el tiempo que debe esperar un cliente
   antes de retransmitir un mensaje. El cliente espera 4 segundos antes de la
   primera retransmisión. Luego duplica el tiempo en espera entre
   retransmisiones, hasta un máximo de 64 segundos.\\[0.2cm]

   No hay un recurso en DHCP para que el servidor DHCP responda a un mensaje
   DHCPDISCOVER con algo como 'no puedo proveer servicio'. Por lo tanto cuando
   un servidor DHCP no es capaz de atender un cliente, simplemente ignora la
   petición. Esto puede pasar porque no tenga IPs disponibles para asignar o
   porque esté configurado para no responder a un(os) cliente(s)
   determinado(s)\\[0.2cm]

\end{frame}

\begin{frame}[fragile]{Evitando colisiones de mensajes}

    DHCP incluye dos mecanismos para disminuir la probabilidad de sobrecargar
    un servidor con mensajes de los clientes. Primero, a un cliente se le puede
    exigir, que retrase su mensaje inicial una cantidad de segundos al azar
    entre 0 y 10 segundos. Otra medida, un cliente DHCP reajustará su tiempo de
    retraso entre retransmisiones de mensaje, con un valor aleatorio de entre
    -1 y 1 para evitar colisión con los mensajes de otro servidor.\\[0.2cm]

\end{frame}

\begin{frame}[fragile,allowframebreaks]{Otros modos de transmisión}

    DHCPREQUEST se emplea en cuatro fases de la vida de un \textit{lease}:

    \begin{itemize}
        \item Durante la selección inicial de un \textit{lease}
        \item Al confirmar la validez de una dirección después de reiniciar
        \item Cuando el lease es renovado desde cualquier punto de su duración
        \item Cuando la renovación de \textit es rechazado cerca del final de
        su duración\\[0.2cm]
    \end{itemize}

    DHCPINFORM es implementado por un cliente que no necesita obtener una
    dirección IP, por ejemplo un servidor con una dirección IP configurada
    manualmente, podría usar un mensaje de este tipo para obtener información
    sobre la red, tal como dirección de servidor NTP, servidores de nombres o
    servidores de impresión. En cualquier caso, el cliente necesita tener su
    propia dirección IP configurada para que pueda asignarla como IP de origen
    en su mensaje\\[0.2cm]

    DHCPRELEASE es empleado para notificar que el cliente va a dejar de usar la
    dirección IP que le había sido asignada, de acuerdo a la especificación, el
    cliente deja de tener una dirección válida y por lo tanto no hay forma de
    entregarla una confirmación de recibido\\[0.2cm]

    DHCPFORCERENEW es el único mensaje que envía el servidor, sin haber sido
    contactado por el cliente. Un servidor puede enviar un mensaje
    DHCPFORCERENEW a cualquier cliente que tenga una dirección asignada por
    DHCP. Es enviado en modo unicast y se emplea para actualizar datos de
    configuración, forzando al cliente que haga un DHCPREQUEST\\[0.2cm]
    
    DHCPLEASEQUERY es un tipo de mensaje de creación reciente y se usa para
    consultar datos sobre un cliente que tenga un \textit{lease} en el servidor
    consultado. Esto sería útil para un 'relay agent' que haya perdido de su
    cache, la información que relaciona un cliente DHCP y un servidor DHCP. En
    tal caso el servidor DHCP revisa su base de datos de \textit{leases} y si
    consigue la data solicitada, la envía en un mensaje DHCPACK al 'relay
    agent' solicitante. Si no existen los datos solicitados, se reponde con un
    mensaje DHCPNAK\\[0.2cm]

\end{frame}

\subsection{Otros intercambios de mensajes DHCP} % (fold)

% subsection Intercambio de mensajes DHCP (end)

    \begin{frame}{Estados del cliente}

    \begin{figure}
    \begin{center}

    \includegraphics[scale=0.60]{images/8-1.png}
    \caption{Maquina de estado finito para un cliente DHCP}
    \label{States-1}

    \end{center}
    \end{figure}

    \end{frame}


    \begin{frame}{Al solicitarse un IP desde una nueva subred}
    \begin{figure}
    \begin{center}

    \includegraphics[scale=0.80]{images/8-5.png}
    \caption{Un cliente DHCP para obtener una dirección desde una nueva subred}
    \label{States-2}

    \end{center}
    \end{figure}
    \end{frame}

    \begin{frame}{Al solicitarse un IP, con dos servidores y el cliente reinicia}
    \begin{figure}
    \begin{center}

    \includegraphics[scale=0.60]{images/8-6.png}
    \caption{Mensajes entre un cliente y dos servidores DHCP, cuando el cliente reinicia}
    \label{States-3}

    \end{center}
    \end{figure}
    \end{frame}

\end{document}
